\documentclass[10pt,letterpaper]{article}

% -------------------- PAQUETES --------------------
\usepackage[margin=2cm]{geometry}
\usepackage{titlesec}
\usepackage{enumitem}
\usepackage[hidelinks]{hyperref}
\usepackage[T1]{fontenc}
\usepackage[utf8]{inputenc}

\setlength{\parindent}{0pt}
\pagenumbering{gobble}

% -------------------- FORMATO --------------------
\titleformat{\section}{\bfseries\large}{}{0em}{}[\titlerule]
\titlespacing*{\section}{0pt}{1em}{0.5em}

\setlist[itemize]{noitemsep, topsep=0pt}

% -------------------- DOCUMENTO --------------------
\begin{document}
	
	\begin{center}
		{\LARGE \textbf{Benjamín Rodríguez Valdez}}\\
		Licenciado en Física — ESFM-IPN\\
		\href{mailto:brodriguezv1184@gmail.com}{brodriguezv1184@gmail.com} \;|\;
		+52~771 33 00 971 \;|\;
		\href{https://github.com/Rodval184}{github.com/Rodval184}
	\end{center}
	
	\vspace{0.5em}
	

	
	% -------------------- PERFIL --------------------
	\section*{Perfil}
	Licenciado en Física con estudios concluidos (egreso enero 2026), con enfoque en \textbf{análisis de datos} y \textbf{educación científica}. Formación sólida en estadística aplicada, programación científica y comunicación técnica. Experiencia académica en procesamiento y visualización de datos, ajuste de modelos, simulaciones Monte Carlo y elaboración de reportes profesionales en \LaTeX. Interés en posiciones \textbf{junior} en análisis de datos, docencia o apoyo técnico--científico.
	
	% -------------------- EDUCACIÓN --------------------
	\section*{Formación Académica}
	\textbf{Licenciatura en Física} \\
	Escuela Superior de Física y Matemáticas (ESFM--IPN), CDMX \\
	Egreso: \textbf{Enero 2026} \\
	Servicio social: En proceso
	
	% -------------------- EXPERIENCIA --------------------
	\section*{Experiencia Académica y Proyectos Relevantes}
	\textbf{Laboratorio de Física Avanzada — ESFM--IPN}
	\begin{itemize}
		\item Análisis y procesamiento de datos experimentales.
		\item Ajustes lineales y no lineales con validación estadística ($\chi^2$).
		\item Visualización de datos mediante Python y Matplotlib.
		\item Elaboración de reportes técnicos y presentaciones académicas.
	\end{itemize}
	
	\textbf{Simulación y Cómputo Científico}
	\begin{itemize}
		\item Implementación de simulaciones Monte Carlo para procesos físicos.
		\item Uso de métodos numéricos y comparación con datos experimentales.
	\end{itemize}
	
	\section*{Experiencia Experimental}
	\begin{itemize}
		\item Caracterización experimental de un amplificador electrónico mediante análisis estadístico.
		\item Análisis cuantitativo de un detector Geiger--Müller, incluyendo la caracterización de la meseta y estudios de atenuación de la radiación.
	\end{itemize}
		% -------------------- HABILIDADES --------------------
	\section*{Habilidades Técnicas}
	\begin{itemize}
		\item \textbf{Análisis de datos:} limpieza, visualización, regresión, estadística básica.
		\item \textbf{Programación:} Python (NumPy, Pandas, SciPy, Matplotlib), C/C++.
		\item \textbf{Herramientas técnicas:} \LaTeX, Beamer.
		\item \textbf{Metodología científica:} análisis crítico de datos y documentación técnica.
	\end{itemize}
	
	% -------------------- IDIOMAS --------------------
	\section*{Idiomas}
	\begin{itemize}
		\item Español: Nativo
		\item Inglés: B1 (lectura técnica)
	\end{itemize}
	
	% -------------------- INTERESES --------------------
	\section*{Intereses Profesionales}
	Análisis de datos · Educación científica · Docencia · Ciencia aplicada
	
\end{document}

